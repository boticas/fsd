% Setup -------------------------------

\documentclass[a4paper]{report}
\usepackage[a4paper, total={6in, 10in}]{geometry}
\setcounter{secnumdepth}{3}
\setcounter{tocdepth}{3}

\usepackage{hyperref}

% Encoding
%--------------------------------------
\usepackage[T1]{fontenc}
\usepackage[utf8]{inputenc}
%--------------------------------------

% Portuguese-specific commands
%--------------------------------------
\usepackage[portuguese]{babel}
%--------------------------------------

% Hyphenation rules
%--------------------------------------
\usepackage{hyphenat}
%--------------------------------------

% Capa do relatório

\title{
	Fundamentos de Sistemas Distribuídos
	\\ \Large{\textbf{Trabalho Prático}}
	\\ -
	\\ Mestrado em Engenharia Informática
	\\ \large{Universidade do Minho}
	\\ Relatório
}
\author{
	\begin{tabular}{ll}
		\textbf{Grupo}
		\\\hline
		PG41080 & João Ribeiro Imperadeiro
		\\
		PG41081 & José Alberto Martins Boticas
		\\
		PG41091 & Nelson José Dias Teixeira
	\end{tabular}
}

\date{\today}

\begin{document}

\begin{titlepage}
    \maketitle
\end{titlepage}

% Resumo

\begin{abstract}
	Este trabalho prático tem por base uma implementação de um sistema de troca de mensagens com persistência e ordenação. Para tal, à semelhança do que feito durante as aulas, foi utilizada a linguagem \textit{Java} (que é orientada aos objetos) por forma a tomar partido de algumas classes já existentes para este tipo de problemas, como por exemplo a classe \textit{Atomix}. De forma geral, o sistema pretendido pode ser descrito como semelhante ao \textit{Twitter} com alguns requisitos extra de correção.
\end{abstract}

% Índice

\tableofcontents

% Introdução

\chapter{Introdução} \label{intro}
\large{
	Neste projeto, de forma muito sucinta, é requerido a elaboração de um sistema distribuído, semelhante ao \textit{Twitter}, baseado na troca de mensagens. Este sistema deve satisfazer alguns requisitos que foram impostos pelo docente desta disciplina. Entre eles destacam-se:
	\begin{itemize}
		\item o sistema deve incluir um conjunto de servidores que se conhecem todos entre si. Estes não devem ter qualquer interação direta com o utilizador. Admite-se ainda a possibilidade de cada um dos servidores ser reiniciado, devendo garantir que o sistema continua operacional depois de todos os servidores estarem novamente a funcionar;
		\item o sistema deve incluir clientes que se ligam a qualquer um dos servidores. Admite-se também que o cliente possa ser reiniciado e posteriormente ligado a um novo servidor.
		\item admite-se que tanto os clientes como os servidores possam fazer uso de memória persistente.
		\item o cliente deve incluir uma interface rudimentar para interagir com o sistema. Nesta deve-se incluir as seguintes funcionalidades:
		\begin{itemize}
			\item permitir a publicação de uma mensagem etiquetada com um ou mais tópicos;
			\item indicar qual a lista de tópicos subscritos;
			\item obter as últimas 10 mensagens enviadas para os tópicos subscritos.
		\end{itemize}
		\item o conjunto de mensagens obtido por cada cliente em cada operação deve refletir uma visão causalmente coerente das operações realizadas em todo o sistema, por esse ou outros utilizadores.
	\end{itemize}
	Após a exposição dos requisitos inerentes a este trabalho prático, os elementos que compõem este grupo deram ínicio ao desenvolvimento do sistema pretendido.
}

\chapter{Análise e Especificação}
    \section{Requisitos}

\chapter{Conclusão}
\large{
	Após a demonstração da abordagem adotada pelo grupo na implementaçao do sistema pedido e, ainda, as adversidades e os desafios que surgiram ao longo da mesma, dá-se por concluída a realização deste projeto. Neste foi possível satisfazer todos os requisitos requeridos bem como acrescentar algumas funcionalidades extra de forma a promover uma melhor interpretabilidade do funcionamento do sistema em causa.
	Para além disso, foi possível aprofundar alguns dos aspetos relativos à componente prática desta unidade curricular, melhorando, assim, a nossa capacidade na implementação de sistemas que abordam esta filosofia computacional.
}

\appendix
\chapter{Observações}
\begin{itemize}
	\item Biblioteca \textit{Atomix}:
	\par \textit{\url{https://atomix.io/}}
	\item Documentação \textit{Java}:
	\par \textit{\url{https://docs.oracle.com/en/java/javase/11/docs/api/index.html}}
\end{itemize}


\end{document}